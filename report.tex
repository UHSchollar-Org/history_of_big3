\documentclass{article}
\usepackage[spanish]{babel}
\usepackage[utf8]{inputenc}
\usepackage{amsfonts}
\usepackage{amsthm}
\usepackage{amsmath}
\usepackage{graphicx}
\usepackage{makeidx}
\usepackage{hyperref}
\setlength{\parindent}{0cm}
\bibliographystyle{IEEEtran}



\begin{document}

\title{Los tres grandes}
\author{Karlos Alejandro Alfonso Rodríguez C411\\Karel Camilo Manresa Leon C412}
\maketitle

\section*{Resumen}

\section*{Introducción}

La electrónica moderna se ha convertido en una de las principales fuerzas impulsoras de la tecnología
en todo el mundo. Desde la invención del transistor a principios de los años
50 hasta la creación de microprocesadores y circuitos integrados en la década de 1960, 
la electrónica ha revolucionado la forma en que las personas se comunican, trabajan, aprenden 
y se entretienen. Los dispositivos electrónicos están presentes en casi todos los aspectos de nuestras 
vidas y se han vuelto esenciales para el funcionamiento de la sociedad actual.

El transistor fue el primer componente electrónico que permitió amplificar y conmutar señales eléctricas
y se convirtió rápidamente en la base para el diseño de circuitos electrónicos. El circuito integrado 
permitió la miniaturización de estos circuitos al integrar varios transistores y otros componentes electrónicos
en un solo chip de silicio, lo que aumentó la velocidad, la eficiencia y la capacidad 
de procesamiento. El microprocesador se convirtió en el cerebro de los sistemas informáticos y ha permitido
la creación de una amplia variedad de dispositivos electrónicos, desde teléfonos móviles hasta 
robots industriales.

Este trabajo de investigación tiene como objetivo explorar la historia y evolución del transistor, 
circuito integrado y microprocesador, tres componentes electrónicos clave que han impulsado 
la revolución electrónica. Se abordará su desarrollo, sus características y sus aplicaciones. 
Además, se explorarán algunos avances tecnológicos recientes en la electrónica que podrían tener 
un gran impacto en la tecnología del futuro.

En definitiva, se trata de una investigación muy relevante que permite comprender el desarrollo de la tecnología
en el siglo XX y la influencia que los componentes electrónicos han tenido en ella. Es por eso 
que este trabajo de investigación busca ser una contribución a la comprensión de la electrónica moderna 
y su importancia en nuestra vida diaria.

\section*{Transistor}

La historia del transistor comenzó en los años 30 del siglo XX, cuando los científicos empezaron a experimentar
con materiales semiconductores. El silicio y el germanio fueron los primeros materiales en ser investigados, 
y se descubrió que estos tenían propiedades de conducción eléctrica que eran diferentes de los conductores metálicos
y los aislantes.

En 1947, un grupo de científicos de los Bell Labs liderado por William Shockley, John Bardeen y Walter Brattain, 
inventaron el primer transistor. Este dispositivo consistía en una estructura de silicio dopado con impurezas que 
permitían el flujo de corriente eléctrica. El transistor reemplazó rápidamente a los tubos de vacío 
que se utilizaban en la época, ya que eran más pequeños, más duraderos y más eficientes.

Los primeros transistores eran grandes y caros, pero en la década de 1950 se produjo un gran avance en la miniaturización
de los mismos. Se descubrió que al reducir el tamaño del transistor, también se reducían los costos de fabricación y se 
mejoraba su rendimiento.

En la década de 1960, se desarrollaron los transistores de efecto de campo (FET), que eran aún más 
pequeños y más eficientes que los transistores bipolares. Los FET permitieron el desarrollo de dispositivos 
electrónicos portátiles como calculadoras y radios de bolsillo. 

En la década de 1970, se produjo otro gran avance en la tecnología de transistores con el desarrollo del 
transistor de unión bipolar de metal-óxido (MOS). Este transistor era aún más pequeño que los FET y consumía 
menos energía, lo que lo hacía ideal para su uso en dispositivos móviles. 

En las décadas siguientes, la miniaturizaciónde los transistores continuó a un ritmo acelerado. En la década de 1980, 
se desarrollaron los transistores de alta velocidad para su uso en la electrónica de telecomunicaciones y en la década
de 1990, se produjo un gran avance con el desarrollo del transistor de efecto túnel (TFET), que permitía una mayor 
eficacia energética y un mayor rendimiento.

Hoy en día, los transistores son fundamentales para la electrónica moderna y se utilizan en casi todos los 
dispositivos electrónicos. \cite{isaacson2019innovadores} plantea que el transistor fue para la era digital, lo que la máquina de vapor había
sido durante la revolución industrial.
La miniaturización de los transistores sigue avanzando a un ritmo rápido, lo que permite la creación de 
dispositivos cada vez más pequeños y eficientes en cuanto a energía.

\section*{Circuito integrado}

Los circuitos integrados (CI), también conocidos como chips, son una pieza clave en la electrónica moderna. 
Los primeros prototipos de CIs se desarrollaron a finales de los años 50 y principios de los 60, 
pero su uso se expandió rápidamente durante la década de 1960. A diferencia de los primeros transistores, 
los circuitos integrados permitieron a los ingenieros combinar varios transistores, diodos y resistencias 
en una sola pieza de silicio.

El primer circuito integrado fue desarrollado por Jack Kilby de Texas Instruments 
y Robert Noyce de Fairchild Semiconductor en 1958. El primer CI contenía sólo un par de transistores, 
pero pronto se desarrollaron chips con docenas, cientos y, finalmente, miles de componentes. 
En 1961, Fairchild Semiconductor lanzó el primer circuito integrado comercial, que contenía 
cuatro transistores y cinco resistencias.

En 1964, la empresa japonesa Toshiba comenzó a producir circuitos integrados a gran escala, 
lo que permitió la producción de chips con cientos de componentes. El primer microprocesador, 
el Intel 4004, se desarrolló en 1971 y contenía 2,300 transistores. Los microprocesadores se convirtieron 
en la base de la computación moderna y permitieron el desarrollo de computadoras personales, teléfonos móviles, 
sistemas de control de automóviles y muchos otros dispositivos.

La tecnología de los circuitos integrados sigue evolucionando hoy en día, con chips más pequeños, más rápidos 
y más eficientes energéticamente. La miniaturización y la integración de la electrónica siguen permitiendo 
el desarrollo de tecnologías nuevas e innovadoras.

\section*{Microprocesador}

La historia de los microprocesadores comienza a mediados de la década de 1960, cuando se desarrollaron 
las primeras calculadoras electrónicas. Estas calculadoras utilizaban circuitos integrados 
para realizar operaciones matemáticas, pero eran muy costosas y requerían un tamaño considerable para ser operadas. 
Fue entonces cuando Ted Hoff, ingeniero de Intel, propuso la idea de crear un circuito integrado 
que pudiera realizar múltiples tareas y que fuera accesible para el consumidor promedio.

En 1971, Intel presentó el primer microprocesador comercialmente exitoso, el Intel 4004. 
Este microprocesador de 4 bits tenía una velocidad de reloj de 740 kHz y fue diseñado para su uso en calculadoras 
y otros dispositivos electrónicos. En 1972, Intel lanzó el Intel 8008, que era un microprocesador de 8 bits 
con una velocidad de reloj de 200 kHz. El Intel 8008 se utilizó en el primer sistema informático personal, 
el MITS Altair 8800.

En 1974, Intel lanzó el Intel 8080, que se convirtió en el primer microprocesador de uso general. 
Tenía una velocidad de reloj de 2 MHz y se utilizó en computadoras como el Altair 8800 y el IMSAI 8080. 
En 1978, Intel lanzó el Intel 8086, que fue el primer microprocesador de 16 bits. El 8086 se utilizó en 
la primera computadora personal, la IBM PC.

A partir de ese momento, los microprocesadores se hicieron cada vez más rápidos y poderosos. En la década de 1980, 
surgieron los microprocesadores de 32 bits, como el Motorola 68000 y el Intel 80386. En la década de 1990, 
los microprocesadores de 64 bits, como el Intel Pentium Pro y el DEC Alpha, comenzaron a ser utilizados 
en sistemas informáticos de alto rendimiento.

Hoy en día, los microprocesadores son la base de la electrónica moderna, y se utilizan en una amplia 
variedad de dispositivos, desde teléfonos inteligentes y tabletas hasta automóviles y equipos médicos.

\subsection*{Primera generación vs actualidad}
Comparación entre los microprocesadores de primera generación y los actuales



\section*{Conclusiones}

%\begin{thebibliography}{2}

%\bibitem{Wal} \textsc{Walter Isaacson},\textit{Los Innovadores}, Editorial: National Geographic Books, 2019.

%\end{thebibliography}

\bibliography{references}

\end{document}